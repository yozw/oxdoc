\chapter{Installation}

\section{Prerequisites}
Since \oxdoc~was written in Java, you should have Java installed on your computer.  
The fact that \oxdoc~is a Java program means that it can in principle be used on 
any operating system, including Windows and Linux. In this section, the installation
process for Windows operating systems will be described. 
For Linux and other operating systems, we will describe the manual installation 
process which is slightly more complicated.

The Java Runtime Environment (JRE) can be downloaded from {\tt www.java.com/getjava}.
Most Linux distributions either have Java pre-installed, or it can be installed from
the installation repositories.

In order to use \LaTeX~generated formulas, a copy of \LaTeX~is required as well.
A free Windows distribution called MiKTeX can be downloaded from {\tt http://www.miktex.org/}.
Alternatively, \oxdoc~can generate mathematical formulas using the JavaScript package
MathML. Generating formulas with MathML does not require any third-party software. 

\oxdoc~uses a program called \dvipng~to generate PNG (Portable Network
Graphics) files from \LaTeX~code.  The full installation of MiKTeX comes with a version
of this programs, but other distributions may not have it readily available.  If you
use a non-full installation of MiKTeX, make sure to select \dvipng~during the
installation. Most Linux distributions have \dvipng~in their installation repositories. For
example, installing \dvipng~on Ubuntu would be done by issuing the following command in a terminal:
\begin{quote}
\tt sudo apt-get install dvipng
\end{quote}

\section{Installation}
\subsection{Installation on Windows 2000/XP using the Setup program}
In order to install \oxdoc, download {\tt setup-0.975alpha.exe} file from
the SourceForge website, run it, and follow the instructions.  
In order to use \LaTeX~formulas, make sure to specify the location of MiKTeX
(or any other \LaTeX~distribution).
The setup program automatically creates a program group in the start menu from
which the \oxdoc~graphical user interface is available. 

\subsection{Manual installation on Windows}
Follow the following steps to manually install \oxdoc:
\begin{enumerate}
\item Unzip the package {\tt oxdoc-xxx-bin.zip} into a
suitable folder. For example, {\tt c:\bs program files\bs oxdoc}.
\item Edit the file
{\tt oxdoc.bat} and alter the {\tt oxdoc} variable in this file. This file looks
as follows:

\begin{quote}
\small \begin{verbatim}
@echo off
set oxdoc=c:\program files\oxdoc
java -jar "%oxdoc%\bin\oxdoc.jar" %1 %2 %3 %4 %5 %6 %7 %8 %9
\end{verbatim}
\end{quote}

The second line in this file has to point to the directory in which the \oxdoc~files
have been unzipped. The same holds for the file {\tt oxdocgui.bat}. 

\item Optionally, copy the files {\tt oxdoc.bat} and {\tt oxdocgui.bat} into a
folder in the Windows search path, e.g. {\tt c:\bs windows}. This way, it is possible to
run \oxdoc~from any folder in the command prompt. 

\item Edit \oxdocxml~in the {\tt bin} directory. This file contains general settings for \oxdoc.
See the Configuration section for more information.
\end{enumerate}

To test whether \oxdoc~works, run the batch file {\tt oxdoc.bat} from the command line.  It should 
display a short description of the program options.

\subsection{Manual installation on Linux}
Follow the following steps to manually install \oxdoc:
\begin{enumerate}
\item Unzip the package {\tt oxdoc-xxx-bin.tar.gz} into a
suitable folder. For example, the \oxdoc~folder in your user directory.

\item Edit \oxdocxml~in the {\tt bin} directory. This file contains general settings for \oxdoc.
See the Configuration section for more information.
\end{enumerate}

To test whether \oxdoc~works, run the script file {\tt oxdoc} from the {\tt bin} directory.  It should 
display a short description of the program options.

