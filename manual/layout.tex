\chapter{Customizing lay-out}

\section{Changing the style sheet}
A large part of the lay-out of the HTML files generated by \oxdoc~is controlled by its
style sheet, {\tt oxdoc.css}. \oxdoc~creates a default
lay-out file if it is not present, but it won't overwrite changes you make
to that file. The {\tt css/} directory of your \oxdoc~installation contains a number of 
standard style sheet files. To choose any of them, just replace the {\tt oxdoc.css} file
by any of the files from that directory. 

\section{Changing the HTML title}
Most browsers will display the title of an HTML page at the top of the window. The title to be used
in the documentation generated by \oxdoc~can be set with the `-windowtitle' option. For example:
\begin{quote}
\tt oxdoc *.ox -windowtitle "My documentation"
\end{quote}

\section{Adding icons}
To make the generated HTML files look a bit nicer, \oxdoc~supports icons. This feature can
be enabled by adding the `-icons` option to the command line, or by adding the following line
to your {\tt oxdoc.xml} configuration file.
\begin{quote}
\tt $<$option name="enableicons" value="yes" /$>$
\end{quote}
This option will generate references to icons in the {\tt icons} subdirectory of your
project folder. The easiest way to get started with this is to copy the standard set of icons 
from the {\tt icons/} subdirectory in your \oxdoc~installation
directory into a new subdirectory called {\tt icons} under your project folder. 

In case you want to use a different set of icons: 
the standard set consists of a number PNG files by the names {\tt xxx.png} and {\tt xxx\_s.png} where
{\tt xxx} is a base file name, e.g. {\tt index.png} and {\tt index\_s.png}. The `{\tt \_s}' stands for
`small'. The dimensions of large icons are 32 times 32 pixels, and the dimensions of the small icons
are 16 times 16 pixels. Although these are the dimensions in the {\tt icons/} subdirectory,
you are in no way obliged to use those specific dimensions. 

