\documentclass[11pt]{article}

\usepackage{fullpage}
\usepackage[sf,bf]{titlesec}
\usepackage{listings}
\usepackage{color} 
\usepackage{times}

\definecolor{CommentColor}{rgb}{0.0,0.5,0.0}
\lstset{language=C,
    basicstyle=\footnotesize\normalfont\ttfamily,
    morekeywords={decl,new,import},
    keywordstyle=\bfseries\color{blue},
    commentstyle=\color{CommentColor},
    stringstyle=\color{red},
    numbers=left,
    emph={SURF},
    emphstyle=\color{blue}} 

\newcommand\oxdoc{{\tt oxdoc}}
\newcommand\oxdocxml{{\tt oxdoc.xml}}
\newcommand\dvipng{{\tt dvipng}}
\newcommand\bs{{\tt\char`\\}}
\begin{document}

\title{\bf User's Manual for {\tt \textbf{oxdoc}}}
\author{Y. Zwols [{\tt yorizwols@users.sourceforge.net}]}
\maketitle\bigskip

\begin{quote}
\small\oxdoc~is a small software package generating documentation in 
HTML format from comments in ox source code. It is inspired by
Sun Microsystems' Javadoc. 
To use \oxdoc, the user needs to write comments in ox sourcecode
in a special format 
and run the \oxdoc~application to extract these comments and
generate a comprehensive HTML (web page) document of the available 
functions and classes in the original code. \bigskip

\oxdoc~is free software and comes with ABSOLUTELY NO WARRANTY.
You are welcome to redistribute it under certain conditions.
See the LICENSE file for distribution details.
\end{quote}



\newpage\section{Installation}

\subsection{Prerequisites}
Since \oxdoc~was written in Java, you should have Java installed on your computer.  
The Java Runtime Environment (JRE) can be downloaded from {\tt www.java.com/getjava}.
The fact that \oxdoc~is a Java program means that it can in principle be used on 
any operating system, including Windows and Linux. In this section, the installation
process for the Windows 2000 or XP operating systems will be described. 
For Linux and other operating systems, we will describe the manual installation 
process which is slightly more complicated.

In order to use \LaTeX~generated formulas, a copy of \LaTeX~is required as well.
A free Windows distribution called MiKTeX can be downloaded from {\tt http://www.miktex.org/}.

\oxdoc~uses a program called \dvipng~to generate PNG (Portable Network
Graphics) files from \LaTeX~code.  The full installation of MiKTeX comes with a version
of this programs, but other distributions may not have it readily available.  If you
use a non-full installation of MiKTeX, make sure to select \dvipng~during the
installation.

\subsection{Installation on Windows 2000/XP using the Setup program}
In order to install \oxdoc, download {\tt setup-0.975alpha.exe} file from
the SourceForge website, run it, and follow the instructions.  
In order to use \LaTeX~formulas, make sure to specify the location of MiKTeX
(or any other \LaTeX~distribution).
The setup program automatically creates a program group in the start menu from
which the \oxdoc~graphical user interface is available. 

\subsection{Manual installation on Windows}
Follow the following steps to manually install \oxdoc:
\begin{enumerate}
\item Unzip the package {\tt oxdoc-xxx-bin.zip} into a
suitable folder. For example, {\tt c:\bs program files\bs oxdoc}.
\item Edit the file
{\tt oxdoc.bat} and alter the {\tt oxdoc} variable in this file. This file looks
as follows:

\begin{quote}
\small \begin{verbatim}
@echo off
set oxdoc=c:\program files\oxdoc
java -classpath "%oxdoc%\bin\oxdoc.jar;%oxdoc%\bin\jregex.jar"  ...
        OxDocCmd %*
\end{verbatim}
\end{quote}

The second line in this file has to point to the directory in which the \oxdoc~files
have been unzipped. The same holds for the file {\tt oxdocgui.bat}. 

\item Optionally, copy the files {\tt oxdoc.bat} and {\tt oxdocgui.bat} into a
folder in the Windows search path, e.g. {\tt c:\bs windows}. This way, it is possible to
run \oxdoc~from any folder in the command prompt. 

\item Edit \oxdocxml~in the {\tt bin} directory. This file contains general settings for \oxdoc.
See the Configuration section for more information.
\end{enumerate}

To test whether \oxdoc~works, run the batch file {\tt oxdoc.bat} from the command line.  It should 
display a short description of the program options.

\subsection{Manual installation on Linux}
Follow the following steps to manually install \oxdoc:
\begin{enumerate}
\item Unzip the package {\tt oxdoc-xxx-bin.tar.gz} into a
suitable folder. For example, the \oxdoc~folder in your user directory.

\item Edit \oxdocxml~in the {\tt bin} directory. This file contains general settings for \oxdoc.
See the Configuration section for more information.
\end{enumerate}

To test whether \oxdoc~works, run the script file {\tt oxdoc} from the {\tt bin} directory.  It should 
display a short description of the program options.










\newpage\section{Using \oxdoc}
Although \oxdoc~is a command line utility at its core, the easiest way to work with it is to use
the graphical user interface (GUI).  If you installed \oxdoc~using the setup program in Windows, 
this interface can be accessed from the Start menu. 

\subsection{Running \oxdoc}
Using \oxdoc~is rather easy. Generating documentation for an ox project
requires running \oxdoc~and specifying the names of the files you
want to generate documentation from. For example, suppose you have a number
of ox files in a folder. From there, run 
\begin{quote}
\tt oxdoc *.ox
\end{quote}

from the command prompt in that folder. \oxdoc~generates a set
of HTML files, of which {\tt default.html} is the project home file. It also
creates a new style sheet file {\tt oxdoc.css}.

It is advisable to specify an output directory for your project. This can
be done by creating a new \oxdocxml~file in your project directory. See 
Configuration for more information on that.

\subsection{Writing documentation}
Now you know how to run \oxdoc, it's time to write some comments
in your code. Documentation comments consist of the normal ox comments,
but instead of using {\tt /*} and {\tt */}, we use
{\tt /**} and {\tt **/}. Documentation comments must be placed 
directly above class definitions and function definitions. For example:

\begin{lstlisting}
/** The multivariate Normal distribution $\mathcal{N}(\mu, \Sigma)$.
An instance of a NormalDistribution class generates realizations of a random
variable $X$ with probability density function 
$$f(x) = |\Sigma|^{-1/2}(2\pi)^{-n/2}
       \exp\left(-\frac{1}{2}(x-\mu)'\Sigma^{-1}(x-\mu)\right).$$

@author Y. Zwols

@example To generate 20 samples from a standard normal distribution, 
the following code can be used:
<pre>
decl Dist = new NormalDistribution(0, 1);
decl Z = Dist.Generate(20);
</pre>
**/
class NormalDistribution {
   decl m_vMu, m_mSigma;
   NormalDistribution(const vMu, const mSigma);
   virtual Generate(const cT);
   virtual Dim();
}

/** Create a new instance of the NormalDistribution class with parameters $\mu$
and $\Sigma$.
@param vMu The mean $\mu$ of the normal distribution
@param mSigma The variance/covariance matrix
@comments The dimension of the multivariate normal distribution is deduced
from the dimensions of the arguments. **/
NormalDistribution::NormalDistribution(const vMu, const mSigma) {
   expectMatrix("vMu", vMu, rows(vMu), 1);
   expectMatrix("mSigma", mSigma, rows(vMu), rows(vMu));
   m_vMu = vMu;
   m_mSigma = mSigma;
}

/** Generate a vector of realizations. The length of the sample is given
by the argument cT.
@param cT Number of samples 
**/
NormalDistribution::Generate(const cT) {
   return rann(cT, Dim());
}

/** The dimension of the multivariate normal distribution.
@comments This is deduced from the arguments given to the constructor. **/
NormalDistribution::Dim() {
   return rows(m_vMu);
}
\end{lstlisting}

This example shows most of the features. Every documentation block is written
between {\tt /**} and {\tt **/} signs; HTML tags can be used, e.g. to add
markup, or include images. Also, documentation blocks are divided into small
sections by {\tt @} commands. For example, parameters can be described in the {\tt @param}
section, and extra comments are given in the {\tt @comments} section. 

Also, the first sentence of the comment block is taken as a summary of the
documentation block.  This first sentence appears in e.g. the project home
page and the methods table.  \oxdoc~recognizes the first sentence by
scanning for a period followed by a white space.  This may have some undesired
effects when a period in the first sentence doesn't indicate the end of a
sentence, e.g. in the sentence 
\begin{quote}
\small\begin{verbatim}
This class implements Dr. John's method. It solves linear equations.
\end{verbatim}
\end{quote}
Here, the part {\tt This class implements Dr.} will be taken as a summary.
This can be avoided by placing {\tt \&nbsp;} (a non-breaking space)
just after {\tt Dr.}:
\begin{quote}
\small\begin{verbatim}
This class implements Dr.\&nbsp;John's method. It solves linear equations.
\end{verbatim}
\end{quote}

Moreover, it is possible to include any HTML tag. This may be useful for
inclusing of images, or adding more intricate mark up.


\subsection{Types of documentation blocks}
There are different types of documentation blocks.  Depending on the position of the blocks, they are
treated differently.  The following blocks are available:

\begin{itemize}
\item {\tt @author} specifies the author of the file. For  usage, see the listing above.

\item {\tt @comments} gives comments. For usage, see the listing above.

\item {\tt @example} gives an example. For usage, see the listing above.

\item {\tt @param} describes a parameter or argument of a function. The 
first word after the {\tt @param} keyword is treated as the name of the 
parameter. More than one parameter can be described by adding more {\tt @param} sections.

\item {\tt @returns} describes the return value.

\item {\tt @see} gives cross references. References have to 
match the exact name of other entities. Multiple references have to be
separated by commas.
\begin{lstlisting}
/** Abstract distribution class
    @see NormalDistribution, UniformDistribution **/
class Distribution { 
   ...
}
\end{lstlisting}

\end{itemize}



\subsection{Using \LaTeX~formulas}
Formulas can be inserted by writing them between single or double dollar (\$) signs. 
For example:
\begin{lstlisting}
<pre>
/** Calculate the OLS estimates for the model $y = X\beta$.
 @returns The OLS estimate $\hat\beta = (X'X)^{-1}X'y$.
 **/
regression(X, y) {
   return invert(X'X)*X'y;
}
</pre>
\end{lstlisting}

Single dollar signs are used for inline formulae, whereas double dollar signs
are used for equations on separate lines, analogously to \LaTeX.  They 
are implemented as {\tt align*} environments.

The way \oxdoc~processes these formulas can be changed.  There are three options:
\begin{enumerate}
\item Plain.  The formulas are copied as-is into the HTML text.  This is not
recommended.

\item \LaTeX.  This uses the \LaTeX~installation on the computer. If you
didn't install \oxdoc~using the setup program, you should specify the
location of {\tt latex} and {\tt dvipng} in \oxdocxml (see also the configuration
section).


\item MathML.  
\end{enumerate}

\subsection{Cross-referencing}
Making cross references within comments is done by placing a symbol between ` signs.  It is important to
specify the whole name of the item to be referenced.  Global functions and classes are identified by
their full names (this is case sensitive!) without arguments, and class methods are identified by the
form {\tt classname::method}. For example, if there is a method {\tt isOk()} in the class {\tt Lumberjack},
this method is referenced to by {\tt `Lumberjack::isOk`}.  The same holds for the {\tt @see} sections.
Note that in {\tt @see} sections, no ` signs should be used.


\subsection{Customizing lay-out}
The lay-out of \oxdoc's output can be controlled by editing the
{\tt oxdoc.css} file in your output directory. Oxdoc creates a default
lay-out file if it is not present, but it won't overwrite changes you make
to that file. 











\newpage\section{Configuration}

\subsection{Location of configuration files}
\oxdoc~is configured by means of the file \oxdocxml.
\oxdoc~looks for this file at two locations:
\begin{enumerate}
\item the directory in which {\tt oxdoc.jar}
is located (e.g. {\tt c:\bs program files\bs oxdoc\bs bin});
\item the current working directory.
\end{enumerate}

Whenever available, settings are loaded from these files in that order.
Parameters set in the current working directory configuration file override
the general settings in the \oxdoc~folder.

It is a good idea to put computer-specific settings in the {\tt bin}
directory and project-specific settings in project directories. 

\subsection{Lay-out of \oxdocxml}
A configuration file looks something like this:

\begin{quote}
\small\begin{verbatim}
<oxdoc>
	<option name="latex"     value="c:\texmf\miktex\bin\latex.exe" />
	<option name="dvipng"    value="c:\texmf\miktex\bin\dvipng.exe" />
	<option name="tempdir"   value="c:\temp\" />
</oxdoc>
\end{verbatim}
\end{quote}

This file specifies values for three options. More option values can be added to
this file as required. See Overview of available settings.

\subsection{Command line configuration}
It is also possible to specify settings through command line arguments
by adding {\tt -parameter=value} to the command line. 
For example,
\begin{quote}
\small\begin{verbatim}
oxdoc -latex=c:\bin\latex.exe *.ox
\end{verbatim}
\end{quote}
specifies a value for the {\tt latex} setting. The names of the
command line parameters correspond exactly to the settings in 
\oxdocxml.

\subsection{\LaTeX~settings}
\oxdoc~uses \LaTeX~in combination with \dvipng to generate PNG
(Portable Network Graphics) files from formulae within comments. In order to get this
working, you'll need a working distribution of \LaTeX~(e.g. MiKTeX
if you're using Windows) and \dvipng~(which comes with MiKTeX). It is then important
to set the paths to the {\tt latex} and \dvipng executables. It is recommended to do this is the \oxdocxml~
file in the {\tt bin} directory of your \oxdoc~installation.

At startup, \oxdoc~checks whether it can find the executables required for \LaTeX~support. If
it can't find one or more of these executables, it automatically turns off \LaTeX~support. In that case,
formulae are literally written in the output. Turning off \LaTeX~support can also be done manually by
setting the {\tt enablelatex} setting to {\tt no}.

It is also possible to specify extra \LaTeX~packages to be included within formulae. This can be done
by specifying the desired packages, separated by commas, in the option {\tt latexpackages}.



\subsection{Overview of available settings}
The following parameters can be set in a configuration file:\medskip

\noindent \begin{tabular}{lp{4.95in}}
\tt outputdir  & Specifies the directory in which \oxdoc~
writes its output. Defaults to the current working directory. \medskip\\
\tt tempdir & Specifies the directory that \oxdoc~can use for temporary files. Defaults to the current working directory.\\
\tt projectname & Specifies the name of the project. This
name will appear in the project home page.\medskip\\
\tt windowtitle & Specifies the title that will appear
in the window caption in your web browser.\medskip\\
\tt imagepath & Specifies the directory in which \oxdoc~
writes its images. Defaults to the specified output directory. \medskip\\
\tt dvipng & Specifies the full path of the executable 
\dvipng. For MiKTeX users, this can be found under the 
{\tt miktex\bs bin} subdirectory of the MiKTeX installation path. \medskip\\
\tt latex & Specifies the full path of the \LaTeX~compiler. 
For MiKTeX users, this can be found under the 
{\tt miktex\bs bin} subdirectory of the MiKTeX installation path. \medskip\\
\tt enablelatex & Turns on or off \LaTeX~support. Possible values:
{\tt yes}, {\tt no}. Default: {\tt yes} if the required executables can be found. \medskip\\
\tt latexpackages & Specifies what \LaTeX~packages should be loaded
for inline \LaTeX~formulas. These packages are loaded in \LaTeX~files 
through the usual {\tt /usepackage{...}} command. Multiple 
packages can be specified by separating them by commas.
\end{tabular}

\end{document}