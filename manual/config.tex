%\newenvironment{options}{\par\noindent\hspace{-6pt}\begin{tabular}{p{1.5in}p{4.95in}}}{\end{tabular}}
%\newcommand{\option}[2]{\tt #1 & #2 \medskip \\}

\newenvironment{options}{}{}
\newcommand{\option}[2]{\par\noindent\hspace{-6pt}\begin{tabular}{p{1.5in}p{4.95in}}\tt #1 & #2 \medskip \end{tabular}}


\chapter{Configuration}

\section{Location of configuration files}
\oxdoc~is configured by means of the file \oxdocxml~and
by command line parameters.
\oxdoc~parses its parameters in the following order:
\begin{enumerate}
\item The \oxdocxml~file in the directory in which {\tt oxdoc.jar}
is located. For example, on Windows operating systems, this file may be found 
in {\tt c:\bs program files\bs oxdoc\bs bin}.
\item The \oxdocxml~file in the current working directory.
\item The command-line parameters.
\end{enumerate}
This means that any parameter that is specified at some stage
in the order above overrides that setting specified in earlier stages.
It is a good idea to put computer-specific settings in the {\tt bin}
directory and project-specific settings in project directories. 

\section{Lay-out of \oxdocxml}
A configuration file looks something like this:

\begin{quote}
\small\begin{verbatim}
<oxdoc>
	<option name="latex"     value="c:\texmf\miktex\bin\latex.exe" />
	<option name="dvipng"    value="c:\texmf\miktex\bin\dvipng.exe" />
	<option name="tempdir"   value="c:\temp\" />
</oxdoc>
\end{verbatim}
\end{quote}

This file specifies values for three options. More option values can be added to
this file as required. See Overview of available settings.

\section{Command line configuration}
It is also possible to specify settings through command line arguments
by adding {\tt -parameter value} to the command line. 
For example,
\begin{quote}
\small\begin{verbatim}
oxdoc -latex c:\bin\latex.exe *.ox
\end{verbatim}
\end{quote}
specifies a value for the {\tt latex} setting. The names of the
command line parameters correspond exactly to the settings in 
\oxdocxml. The yes/no options ({\tt -icons} and {\tt -showinternals})
are specified without parameter. Also, if the parameter contains spaces,
one should put the value between double quotes. For example:
\begin{quote}
\small\begin{verbatim}
oxdoc -latex "c:\program files\miktex\bin\latex.exe" -formulas latex 
      -icons -showinternals *.ox
\end{verbatim}
\end{quote}

\subsection{\LaTeX~settings}
\oxdoc~uses \LaTeX~in combination with \dvipng~to generate PNG
(Portable Network Graphics) files from formulas within comments. In order to get this
working, you'll need a working distribution of \LaTeX~(e.g. MiKTeX
if you're using Windows) and \dvipng~(which comes with MiKTeX). It is then important
to set the paths to the {\tt latex} and \dvipng executables. It is recommended to do this is the \oxdocxml~
file in the {\tt bin} directory of your \oxdoc~installation.

At startup, \oxdoc~checks whether it can find the executables required for \LaTeX~support. If
it can't find one or more of these executables, it automatically turns off \LaTeX~support. In that case,
formulas are literally written in the output. Turning off \LaTeX~support can also be done manually by
setting the {\tt formulas} setting to {\tt plain} or to {\tt mathml}.

It is also possible to specify extra \LaTeX~packages to be included within formulas. This can be done
by specifying the desired packages, separated by commas, in the option {\tt latexpackages}.
The specified packages will be included into every \LaTeX~formula by means of the {\tt \bs usepackage}
command.



\section{Overview of available settings}
The following parameters can be set on the command line or in a configuration file. 

\subsection{Options for input}
\begin{options}
\option{include}{Specifies include search path. On Windows, the paths should be separated by 
semicolons (;). On Linux, they should be separated by colons (:). 
}
\end{options}

\subsection{Options for output}
\begin{options}
\option{formulas}{Chooses the way formulas are handled. Possible values: {\tt latex}, {\tt mathml}, {\tt plain}. 
Default: {\tt latex} if the required executables can be found, {\tt plain} otherwise.}

\option{icons}{Specifies whether the generated HTML files should contain references to icons. 
Possible values: {\tt yes}, {\tt no}. If set to {\tt yes}, then the generated HTML files
will contain references to icons in the {\tt icons/} folder. See Section \ref{sec:icons} on how to
implement these icons.}

\option{outputdir}{Specifies the directory in which \oxdoc~ writes its output. Defaults to the current working directory.}

\option{projectname}{Specifies the name of the project. This
name will appear in the project home page.}

\option{showinternals}{Specifies whether internal methods and fields (i.e. methods and fields that are private, protected, or marked with the @internal clause) 
are included in the documentation. Possible values: {\tt yes}, {\tt no}. See \ref{sec:showinternals}}

\option{windowtitle}{Specifies the title that will appear
in the window caption in your web browser.}

\end{options}

\subsection{Options for third-party utilities}
\begin{options}

\option{dvipng}{Specifies the full path of the executable 
\dvipng. For MiKTeX users, this can be found under the 
{\tt miktex\bs bin} subdirectory of the MiKTeX installation path.}

\option{imagepath}{Specifies the directory in which \oxdoc~writes the images that
represent \LaTeX~formulas. Defaults to the subdirectory `images' in the specified output directory.}

\option{imagebgcolor}{The background color for LaTeX images. Possible values: either an
HTML color in the form {\tt \#RRGGBB}, where RR, GG and BB are hexadecimal representations
of the intensity of the red, green, and blue components, or {\tt transparent}.
Examples: {\tt \#0000FF} for bright blue. The use of this is to make sure that the 
background color of the \LaTeX~generated formula images coincides 
with the background color of the HTML pages. The default, {\tt transparent}, 
generates PNG files that have transparent backgrounds. Unfortunately, not every
web browser supports such transparent backgrounds, hence this option.}

\option{latex}{Specifies the full path of the \LaTeX~compiler. 
For MiKTeX users, this can be found under the 
{\tt miktex\bs bin} subdirectory of the MiKTeX installation path.}

\option{latexpackages}{Specifies what \LaTeX~packages should be loaded
for inline \LaTeX~formulas. These packages are loaded in \LaTeX~files 
through the usual {\tt \bs usepackage\{...\}} command. Multiple 
packages can be specified by separating them by commas.}

\option{tempdir}{Specifies the directory that \oxdoc~can use for temporary files. Defaults to the current working directory.}

\end{options}
